\section{Essence} \label{sec:essence}
For the process of project development, we have chosen to work with Essence. 
Previous semesters we have worked with an agile approach inspired by Scrum, however, this semester we are attempting to apply the Essence approach.
The basic idea of Essence is to encourage diverse thinking in the team, even though all members of our team share a similar background as bachelors in Software.
\\\\
Essence uses two strategies to support value creation:
\begin{itemize}
    \item \textit{A systematic use of diverse viewpoints.} 
    Values, views, and roles are used frequently in Essence. 
    By using different views and roles to represent problems and solutions, Essence tries to facilitate a range of viewpoints on how a problem needs to be understood and solved.
    \item \textit{A focus on idea maturation more than idea generation.}
    Essence applies the concept that ideas develop over time and tries to stimulate the team to evaluate and refine ideas \cite{Essence}.
\end{itemize}

\subsection{Four variants of innovation}
% Overvej om det er nødvendigt
Essence tries to support innovation in software development, and hereby it defines four different variants of innovation, which are: \cite{Essence}:
\begin{itemize}
    \item \textit{Product innovation} is new or radically changed software products or services. 
    \item \textit{Process innovation} is software solutions offering the user new or radically improved ways to produce products or services.
    \item \textit{Project innovation} is fitting software solutions from earlier projects into new application domains
    \item \textit{Paradigm innovation} is about software solutions coupled with changes in the mental model of what a business is, who the users are or what the market is.
\end{itemize}

\subsection{Paradigms}
There are two well-established software development paradigms: the document-oriented paradigm, which we know from the waterfall approach and agile paradigm which we know from for example extreme programming and Scrum.
The author of Essence considers the new emerging paradigm called the pragmatic paradigm.
\\\\
The document-oriented paradigm portrayed software developers as being document-oriented. 
The requirements are static and allow for a top-down waterfall approach to software development, which pays small attention to creativity and innovation.
\\\\
The agile paradigm sees software development as user-oriented.
Requirements are presumed dynamic as customers learn about options and constraints within the course of the project.
This leads to incremental software development.
\\\\
The pragmatic paradigm is problem-oriented. 
Systems are becoming more complex and it is more difficult to separate systems from each other. 
The amount of data, software libraries and hardware components available is steadily increasing, leading to hypercomplex software projects.
Hypercomplexity is a degree of complexity that makes it impossible to make rational decisions within a reasonable time constraint.
The most important features of this paradigm are that requirements are not completely known when the projects start.
Ideas evolve in the process of the project, and during the project the requirements for the project are negotiable \cite{Essence}. 

\subsection{Core concerns}
All software projects involve these four core concerns:
% Jeg tror ikke at dette afsnit er nødvendigt.

\begin{itemize}
    \item Do we know what to build?
    \item Do we understand the solution?
    \item Do we understand the problem?
    \item Should we pivot or persevere?
\end{itemize}

\subsection{Team organization}
\label{sec:team-organization}
Within the team organization, in Essence, roles are used to create heterogeneity in teams, to ensure diverse points on views and to ensure cohesion despite diversity.
The focus of these roles is to increase learning with personal interaction by sharing insights and experiences.
The roles also ensure that the team understands the problem domain, and see potentials in the technology domain.
\\\\
As a rule of thumb, the roles are persistent meaning that a member will have the same role for the duration of the project.
The roles in Essence are compatible with agile software development, making it possible to combine Essence with other processes like Scrum.
\\\\
There are four roles in essence:
\begin{itemize}
    \item Child
    \item Responder
    \item Challenger
    \item Anchor
\end{itemize}
The role of \textit{Child} can ask any questions and make propositions that are opposite of previous decisions.
The rest of the team is not allowed to criticize the child, but they are however allowed to ignore the person's suggestions.
The child is one of the main sources of ideas and other perspectives on the project.
Outsiders are also allowed to take this role.
\\\\
\textit{Responders} are the developers in the team, and they are usually the majority within the team.
Responders work closely together with the \textit{Challenger}, so that the most important features are developed first.
\\\\
\textit{Challenger} is the customer or customer representative. 
The challenger can be compared to the \textit{Product Owner} in Scrum.
This role formulates and explains the Challenge, prioritizes features and accepts the solutions.
There can be more than one Challenger, but if there are they must agree on the product vision.
\\\\
\textit{Anchor} is the one responsible for leading evaluations but does not decide the consequences.
If necessary, the anchor can intervene and remove threats to the team's ability to develop ambitious responses. 
A potential threat could be something that results in productivity issues.
