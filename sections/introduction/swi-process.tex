\section{Essence}
For the process of the project development, we have chosen to work with essence. 
Previous semesters we have worked with an agile scrum approach, but this semester we are working with Essence.
The basic idea of Essence is to stimulate diverse thinking in the team, even though our team shares a similar background of a bachelors degree in Software.
\\\\
Essence uses two strategies to support value creation:
\begin{itemize}
    \item \textit{A systematic use of diverse viewpoints.} 
    Values, views and roles are used frequently in Essence. 
    By using different views and roles to represent problems and solutions, Essence try to facilitate a range of viewpoints of how a problem needs to be understood and solved.
    \item \textit{A focus on idea maturation more than idea generation.}
    Essence applies the concept that ideas develop over time and tries to stimulate the team to evaluate and refine ideas \autocite{Essence}.
\end{itemize}

\subsection{Four variants of innovation}
There are different kinds of innovation which are important within Essence, as the roles considers the different types of innovation. 
Essence defines four different variants of innovation, which are: \autocite{Essense}
\begin{itemize}
    \item \textit{Product innovation} is new or radically changed software products or services. 
    \item \textit{Process innovation} is Software solutions offering the user new or radically improved ways to produce products or services.
    \item \textit{Project innovation} is fitting software solutions from earlier projects into new application domains
    \item \textit{Paradigm innovation} is about software solutions coupled with changes in the mental model of what a business is, who the users are or what the market is.
\end{itemize}

\subsection{Paradigms}
There are 2 well established software development paradigms, the document oriented paradigm and agile paradigm.
The author of Essence considers a new emerging paradigm called the Pragmatic paradigm.
\\\\
The document oriented paradigm showed software developers as being document oriented. 
The requirements were static and allowed for a top down waterfall approach to software development, which pays small attention to creativity and innovation.
\\\\
The agile paradigm saw software development as user-oriented.
Requirements were presumed dynamic as customers learns about options and constraints within the course of the project.
This leads to incremental software development.
\\\\
The pragmatic paradigm is problem-oriented. 
______ er kommet hertil
Central elements in this paradigm is that the requirements are not completely known when projects start. 
The 


- Document oriented
- Agile 
- Pragmatic / problem oriented

We use the pragmatic because our problem is hypercomplex... + mere

\subsection{Core concerns}

- Do we know what to build?
- Do we understand the solution
- Do we understand the problem?
- Should we pivot or perservere

\subsection{Team organization}
We use this process so that team members will constantly reflect on whether the project makes sense...

Design principles

Why use roles?

Focus of roles

Roles
- Child
- Responder
- Challenger
- Anchor

\subsection{How it used}