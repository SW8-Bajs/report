\section{Mobility}\label{sec:mobility}
As a part of the study regulation it is emphasized that the system needs to be mobile.
The essence of the project idea as described in \autoref{sec:projectidea}, is that the players are mobile and easily can use their smartphone and virtual reality headset as the viewable device to view the game.
To achieve this form of mobility it requires us to handle wireless communication between users and by calculating the location of the players by the use of sensors.
\\\\
We also aim towards having the entire game as being mobile, which means that you would be able to set up the game at different locations.
This can however be more difficult, as some type of location-based service has to be used, as well as the need for connection between users to communicate.
Setting up the game at other locations also requires more knowledge about the game from the user, than the regular player otherwise would know, as the person has to set up the sensors at correct positions to measure the distance between the components.