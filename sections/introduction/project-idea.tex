\section{Initial project idea}\label{sec:projectidea}
In the following section, the initial idea for the project will be presented as well as some features that the game could include.
The initial idea for this project is to create a location-based game using virtual reality (VR), based on a proposal by our supervisor Brian Nielsen who suggested using Pozyx for making a system that could track a soccer ball and players throughout a game and generate a replay of the match afterwards.
In our initial idea, the game will have two teams will compete against each other to score the most goals using a ball in a soccer-like fashion.
Each player will be equipped with a smartphone-based virtual reality goggles, and these will display the playing field from a top-down 2D view. 
To achieve this, each player's position needs to be tracked as well as where the ball is located on the field.
In the top-down view, each player needs to see the positions of the other players and the ball.
They also need to see their position on the playing field and where the goals are.
The players should be able to set a number of goals they need to score to win before beginning the game. 
\begin{figure}[H]
    \centering
    \includegraphics[width=0.6\linewidth]{Spil.PNG}
    \caption{An illustration of game idea}
    \label{fig:game_illustration}
\end{figure}
\noindent
In \autoref{fig:game_illustration}, a mock-up of the playing field for the game is shown.
There are goals at each end of the field, and the teams score goals by getting the ball between the goalposts.
An alternative version of the game was also discussed where, instead of goals at each end of the field, there would be virtual goal zones seen in the game which the teams need to bring the ball into.
These zones could even change locations as the game progressed.

\subsection{Technical requirements for the project}
Various hardware will be required to realize the vision of the game.
First of all, each player must have a goggles to hold their smartphone such that they can view a virtualized version of the playing field while playing.
In order for the data to be synchronized between the players, they will also need to be equipped with a positioning device, which can transmit their location to the other players.
This transfer of information will require a networking solution so that the virtualized playing field is synchronized between the players.

\subsection{Problems to consider}
The initial project idea proposes some problems that will need to be solved for the game to work.
We will need to consider which technologies to use for the development of the visual aspect of the game which should show a top-down view for each player. 
As we do not have experience creating VR-based games, it would be preferable if it is not necessary to have to build it from scratch.\\
Additionally, hardware that can track the positions of players and the ball is needed.
This positional data must be accurate and update quickly such that the players do not run into each other, causing the game to not work as intended.
Another problem to consider is how the ball should be displayed in a 2D view.
For the players to be able to find the ball on the field, it either has to be quite large to make it easier to find from the top-down view, or the game will need some metric to display how far the ball is from the ground.
The game will also need to track when the ball has crossed the goal line and then give feedback to the players.
Another problem to solve is how to keep the positional data synchronized across all the players' devices, as it will be difficult to play the game if all players are not in the same game state.
