\section{Limiting the amount of UDP transmissions}
As the game was being tested we encountered a problem with the amount of UDP packages that were being transmitted.
Initially there were no limitations to the number of packages sent, and the server would transmit as many as it possibly could.
This would overburden the internet and could cause it to crash.
A limit had to be set for how many packages were able to be transmitted per second.
Since Pozyx can send updates on tags at a rate of approximately 60 per second, it would be preferable to have the limit at close to that amount \cite{pozyx-Performance}.
The limit could even be a bit higher than 60, since the updates are being transmitted with UDP which means that there is no assurance that each package arrives.
Eventually, a limit of 70 packages per second was chosen based on the Pozyx limit and the chance of packages not arriving.
\\\\
\begin{lstlisting}[caption={Implementaion of the limit on the amount of packages that can be sent per second}, captionpos=b,language=C,label={lst:package_limitation}]
    self.time_now = time.time()
    if(self.time_prev != None):
        self.bytesAheadOfSchedule -= self.ConvertSecondsToBytes(self.time_now - self.time_prev)
    self.time_prev = self.time_now

    self.bytesAheadOfSchedule += 7
    if(self.bytesAheadOfSchedule > 0):
        time.sleep(self.ConvertBytesToSeconds(self.bytesAheadOfSchedule))

    # Send message to all clients listening on the multicast_group
    self.sendto(message, self.multicast_group)
\end{lstlisting}
The code in \autoref{lst:package_limitation} is from the \texttt{send} function of the UDP socket.
This code is run each time a UDP package is sent.
Initially the current time is stored, and the time difference between the current time and the time when the previous package was sent is converted to bytes.
The variable \texttt{bytesAheadOfSchedule} keeps track of whether or not too many bytes are being sent per second.
Since the size of the packages sent with UDP is always the same size, being 7 bytes, \texttt{bytesAheadOfSchedule} is incremented with 7 for each package sent \autoref{sec:sprint1-networking}.
If \texttt{bytesAheadOfSchedule} is above 0, too many packages are transmitted and the server should wait a bit before sending the next package using \texttt{time.sleep()}.
By doing this for each call to the \texttt{send} function, the number of packages sent per second can be limited to 70 per second.
\\\\
\begin{lstlisting}[caption={function for converting seconds to bytes and bytes to seconds}, captionpos=b,language=C,label={lst:conversion_functions}]
    def ConvertSecondsToBytes(self, numSeconds):
        return numSeconds*self.maxSendRateBytesPerSecond

    def ConvertBytesToSeconds(self, numBytes):
        return float(numBytes)/self.maxSendRateBytesPerSecond
\end{lstlisting}
\noindent
The conversion functions between bytes and seconds are seen in \autoref{lst:conversion_functions}.
\texttt{maxSendRateBytesPerSecond} is defined as the the maximum number of packages that can be sent per second, 70 in this case, times the size of the package in bytes which is 7.
The conversion is then simply a case of either dividing the number of bytes provided with \texttt{maxSendRateBytesPerSecond} which gives the amount of seconds the system should sleep.
\texttt{ConverSecondsToBytes} is used for figuring out the number of bytes \texttt{bytesAheadOfSchedule} should be decremented with.
The parameter passed to this function is the number of seconds that passed between the current and previous call to the \texttt{send} function.
