\section{Limiting the amount of UDP transmissions}
As the game was being tested we encountered a problem with the amount UDP packages that were being transmitted.
Initially there were no limitiations to the amount of packages sent, and the server would transmit as many as it possible could.
This would overburden the internet and could cause it to crash.
A limit had to be set for how many packages was able to be transmitted per second.
Since Pozyx is able to send updates on tags at a rate of approximately 60 per second, it would be preferable to have the limit at about the same as that.
The limit could even be a bit higher than 60, since the updates are being transitted with UDP which means that there is no assurance that each package arrives.
Eventually a limit of 70 packages per second was chosen.
\\\\
\begin{lstlisting}[caption={Implementaion of the limit on the amount of packages that can be sent per second}, captionpos=b,language=C,label={lst:package_limitation}]
    self.time_now = time.time()
    if(self.time_prev != None):
        self.bytesAheadOfSchedule -= self.ConvertSecondsToBytes(self.time_now - self.time_prev)
    self.time_prev = self.time_now

    self.bytesAheadOfSchedule += 7
    if(self.bytesAheadOfSchedule > 0):
        time.sleep(self.ConvertBytesToSeconds(self.bytesAheadOfSchedule))

    # Send message to all clients listening on the multicast_group
    self.sendto(message, self.multicast_group)
\end{lstlisting}
The code from \autoref{lst:package_limitation} is from the send method of the UDP socket.
This is code is run each time an UDP message is sent.
Initially the current time is stored, and the time difference between the current time and the time when the previous message was sent is converted to bytes.
The variable \texttt{bytesAheadOfSchedule} keeps track of if too many bytes are being sent per second.
Since the size of the packages sent with UDP are always the same size, being 7 bytes, \texttt{bytesAheadOfSchedule} is incremented with 7 for each package sent.
If \texttt{bytesAheadOfSchedule} is above 0, too many packages are transmitted and the server should wait a bit before sending the next message using \texttt{time.sleep()}
\\\\
\begin{lstlisting}[caption={function for converting seconds to bytes and bytes to seconds}, captionpos=b,language=C,label={lst:conversion_functions}]
    def ConvertSecondsToBytes(self, numSeconds):
        return numSeconds*self.maxSendRateBytesPerSecond

    def ConvertBytesToSeconds(self, numBytes):
        return float(numBytes)/self.maxSendRateBytesPerSecond
\end{lstlisting}
The conversion functions between bytes and seconds are seen in \autoref{lst:conversion_functions}