\section{Discussion}
The following section will discuss various aspects of this project, investigating what has been accomplished and what could be improved.
The section will look at the process used for development, the overall theme of mobility in relation to the project, the final design of the application, some considerations about using UDP and TCP, the testing of the system, and finally compare the end product to the previously established MVP.

\subsection{Reflecting on the process}
The process used during the development of this project was described in \autoref{sprint1:ourprocess}, blending elements of Scrum and Essence.
Development was segregated into five sprints, defining the different iterations of the system.
These sprints served as a guideline for when to have sprint retrospectives and enact changes in the process.
Each retrospective would cause slight alterations to the process.
\\\\
Initially, tasks were estimated based on planning poker.
However, we found it difficult to properly estimate a lot of the tasks, as this project concerned itself with an area of expertise we were not familiar with.
This led to the inclusion of an Essence based version of estimating in sprint 1.
This version of estimation was also scrapped, as described in \autoref{sec:sprint2-conclusion}, due to it seeming biased towards shorter tasks and estimation not being accurate.
This led to the method of simply having a group-based discussion about the DoD of a task, and ensuring everyone understood it.
This DoD based task assignment worked well for the final sprints.
It avoided spending unnecessary time on planning poker on top of discussing tasks and made the developers more productive.
\\\\
Another aspect improved by the retrospectives was the stand-up meetings.
At the beginning of the project these tended to be unfocused and not provide the necessary information, and were not conducted daily.
This led to the introduction of a certain set of questions, defined in \autoref{sec:sprint2-conclusion}, which helped to ensure the developers stayed focused and provided the necessary information to facilitate productivity.
Along with the questions, the stand-up meetings were changed to be conducted daily, which further facilitated the developers staying focused.
\\\\
\todo{pivoting and essence}

\subsection{Reflecting on mobility}
\todo{VI SKAL HAVE INTRODUCERET NOGET MOBILITY. NÅR DET ER GJORT SKAL DER VÆRE NOGET REF TIL DET HER.}
This system achieves the goal of mobility in two separate ways.
The main aspect of the system as a whole is the game itself.
This game is developed for mobile devices and achieves mobility in the sense that the application can be used by clients in many different locations, rather than being restricted to a certain location.
One issue with this, however, is the Python-based host application.
This application is currently developed for use on a computer, meaning it is not as mobile as the game counterpart.
It is not fully restricted to a certain location, however, as it can be executed on a laptop which can be moved to the location of the playing field.
\\\\
The game requires an internet connection to be played.
This could also limit mobility, but during the test described in \autoref{sec:sec:evaluatin_test} the game was played outdoors using a mobile network.
Doing so facilitates both indoor and outdoor play, further improving the mobility of the system in the sense that it can be used in different conditions.
Another way this system achieves the mobility aspect is that when in use, the game requires players to be mobile.
While players are mobile when using the system, the Pozyx system is used to provide a service in regards to location information which is communicated wirelessly by the underlying network protocol.

\subsection{The final design}
\todo{do this when design intro is done and final screenshots have been taken}
no lobby, bad

\subsection{UDP and TCP for the protocol}
When the socket type to be used for the program was initially chosen in \autoref{sec:sprint1-networking}, a large emphasis was put on the rate at which we could send messages.
The sending rate also affected a lot of the other decisions made for the networking protocol, such as keeping messages as short as possible.
The goal was to facilitate sending as many messages as possible.
However, once testing began, it became obvious that this might not have been the fully correct approach.
While running the host with mock data, it was possible to more or less crash our connections.
\\\\
In order to fix this, we implemented some message throttling, described in \autoref{sec:sprint5-transmissionlimit}, based on the maximum amount of updates from the available Pozyx system.
Keeping this limiting of messages in mind, the aspect of sending messages rapidly loses some of its importance, leading to some of our earlier considerations not being optimal.
This might have led us to implement a TCP aspect earlier in the project, avoiding having to spend some of the time attempting to implement UDP acknowledgments.

\subsection{Testing the system}
The first full-scale test of the system and all its components took place on May 7th, as described in \autoref{sec:initial-test}.
This was late in the overall process of the semester, and showed certain flaws with the system that should have been caught earlier, as development was being done on the different parts of this system.
The overall purpose of this test was initially to document flaws with the usage of the program, but it turned into a test to ensure the components worked together.
This obviously delayed the testing of the actual running of the program.
\\\\
This was not ideal.
The preparation for the test was not sufficient, and it led to wasting the time of some of the developers that could have been spent performing other tasks while the components were tested together.
An obvious solution to this problem would be to do more integration testing as the project was progressing and new components became ready for testing.
A complication with this is that this system uses Pozyx for positional data, and due to the corona outbreak, the group members were prohibited from meeting physically.
One person still had access to the required Pozyx components however and could have performed the integration tests.
\\\\
The person with access to the Pozyx system did test the system a few times, however, due to miscommunication, this was mainly done with mock data rather than actual data for a setup used to play the game.
This was the main reason we thought it would be possible to test the running of the program on May 7th, as we thought the program was ready.
The communication should have been more clear, and more care should have been put into continuous testing of the system.
These issues led to the necessity of conducting a second test, described in \autoref{sec:sec:evaluatin_test}, for which more care was put into preparation to ensure the generation of results.

\subsection{Comparing the final product to the MVP}
In \autoref{sec:mvp} an MVP was defined for a viable game solution, split into functional and non-functional requirements.

\subsubsection{Functional requirements}
The functional requirements defined the core features of the game.
The first three requirements are defined as starting the game, connecting clients and the server, and users viewing their position.
All of this is accomplished through the network protocol defined in \autoref{app:network}, specifically TCP message 3 for game start and UDP message 0 to update player positions.
Connecting players is done through the lobby screen based on IP.
The next few requirements deal with actually playing the game, requiring the player to be able to view a playing field, goals, the players, the ball, and current score, as well as being able to win a game based on their current score.
This has all been accomplished in the Unity game, as shown in\todo{ref til sprint 5 conclusion en gang.}.
Finally, the game should be dynamic, allowing for an arbitrary number of players and goals to win as well as different sizes of playing fields.
This is partly accomplished.
The game supports an arbitrary number of goals to win, however, in terms of players there is currently a maximum capacity of four players in a single game.
This is communicated by the host to the clients through TCP message 6 in \autoref{app:network}.
The playing field scales based on the user input, meaning that requirement is supported, but fields have to be shaped like a rectangle or a square in order to support the current implementation of generating new goal zones.

\subsubsection{Non-functional requirements}
The non-functional requirements would ideally be evaluated through a usability test.
Such a test could not be conducted properly due to the circumstances.
Instead, a more unstructured test was conducted in \autoref{sec:sec:evaluatin_test} with the group.
However, the non-functional requirements state that the game should be pleasant to look at and not cause discomfort for players, and there should be enough updates to represent the real positions of the players.
None of the testers experienced nausea or discomfort during the playing of the game, indicating that the first requirement was likely achieved.
There were only six testers however, which is likely not enough testers to determine this completely.
The testers could have been outliers, or there could be possible outliers who would find the playing of the game uncomfortable.
As explained in \autoref{sec:evaluatin_test}, there were some slight issues in terms of updates when playing the game outside.
Occasionally, updates would stop being sent for a period of several seconds.
This caused the game to be interrupted.
\todo{Den her sætning herunder. Er det godt nok sådan her? Eller er det at vi blamer tags?}
The reason for this issue has not been pinpointed, and it could have been caused by an unstable connection to the mobile network.
Another reason for this could be players turning the tag upside down since Pozyx works the best when it is turned properly\cite{pozyx-AnchorHeights}, or simply the signal being obscured if the tag is kept in the pocket of a player.
In order to combat these issues, we implemented linear interpolation as described in \tocdo{autoref til lerp}.

\section{Scalability}
During the semester, the game was mainly tested with mock data due to the lockdown.
In the actual tests of the system, as described in \autoref{sec:evaluatin_test}, two of the Pozyx tags stopped working after a firmware update and could not be used for getting positional data.
This also means that the game has only been tested with two and four players.
However, the game should easily be able to scale up to an arbitrary amount of players given that the hardware is available.
The biggest problem with scalability is in terms of larger fields since the anchors only have a range of up to approximately 20 meters. To have a field larger than this, the game will need to support placing anchors along the edge of the playing field, rather than just in the corners.
This is currently not supported and will require some refactoring of the system since the field is currently generating the field based on the four anchors, which would need to be changed to a system where it is told where the four corner anchors are.
Additionally, this may lead to some increasingly irregularly shaped playing fields, which may require another algorithm to generate the goal positions.


\section{Evaluation of Essence}
In \autoref{sprint1:ourprocess} a brief presentation was made of the process and the team organization in the Essence process model.
Throughout the semester, this process was readjusted to better suit the needs and workflow of the group.
One of the main things that was used from Essence was the roles.
While Essence includes things like potential and filters for the pre-project, these were not introduced until the pre-project phase of the semester had passed, and was deemed less meaningful to use at these stages.\\
Additionally, since not all group members took the software innovation class where Essence was presented, it took a lot of work to relay the information since the remaining group members would essentially have to learn the curriculum of another course on top of the ones they were already following.
\\\\
Our experience with the Essence roles was that it was quite similar to the roles present in the well-known Scrum approach.
The challenger, in Essence, has a similar role as a product owner, the anchor as the scrum master, and the responder corresponds to being a member of the development team.
The primary addition that Essence has to roles is the addition of the child role.
This role was supposed to be a fluctuating role which allowed group members to come with feedback and encourage divergent thinking.\\
However, since a lot of the pre-project parts of Essence were skipped, the child role did not make a lot of sense and quickly ended up becoming a joke to suggest silly ideas rather than a useful tool.
\\\\
Had we known about all parts of Essence before starting the project, there is a large possibility that it would have been significantly more useful, and that going through the pre-project models would have helped in increasing the quality of the final system.
This assumption is based on the theory that having a clearer idea about the direction and encouraging more in-depth discussions before making choices would have resulted in different choices along the way.
\\\\
One of the primary motivators behind using Essence over Scrum is the focus on innovation, where it is encouraged to shift direction along the way, while still staying within a pre-defined problem domain.
While we attempted to keep this approach in mind, it was difficult due to the strict time constraint that prohibited us from making major changes in the focus, since we would still have to consider the study regulations and whether we could change focus and still be done in time for every possible pivot.
An example of where we decided to pivot despite the cost involved was in \autoref{sec:update-network-protocol} where the networking protocol was almost entirely reworked from being purely UDP to being a mix of UDP and TCP.
