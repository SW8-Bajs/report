\section{Reflecting on mobility}
\todo{VI SKAL HAVE INTRODUCERET NOGET MOBILITY. NÅR DET ER GJORT SKAL DER VÆRE NOGET REF TIL DET HER.}
This system achieves the goal of mobility in two separate ways.
The main aspect of the system as a whole is the game itself.
This game is developed for mobile devices and achieves mobility in the sense that the application can be used by clients in many different locations, rather than being restricted to a certain location.
One issue with this, however, is the Python-based host application.
This application is currently developed for use on a computer, meaning it is not as mobile as the game counterpart.
It is not fully restricted to a certain location, however, as it can be executed on a laptop which can be moved to the location of the playing field.
\\\\
The game requires an internet connection to be played.
This could also limit mobility, but during the test described in \autoref{sec:evaluatin_test} the game was played outdoors using a mobile network.
Doing so facilitates both indoor and outdoor play, further improving the mobility of the system in the sense that it can be used in different conditions.
Another way this system achieves the mobility aspect is that when in use, the game requires players to be mobile.
While players are mobile when using the system, the Pozyx system is used to provide a service in regards to location information which is communicated wirelessly by the underlying network protocol.
