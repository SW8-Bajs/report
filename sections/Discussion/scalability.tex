\section{Scalability}
During the semester, the game was mainly tested with mock data due to the lockdown.
In the actual tests of the system, as described in \autoref{sec:evaluatin_test}, two of the Pozyx tags stopped working and could not be used for getting positional data.
The reason for this could not be pinpointed during the test.
This also means that the final game has only been tested with two players using Pozyx and four players using mock data.
However, the game should easily be able to scale up to an arbitrary amount of players given that the hardware is available.\\
The biggest problem with scalability is in terms of larger fields since the anchors only have a range of up to approximately 20 meters.
To have a field larger than this, the game will need to support placing anchors along the edge of the playing field, rather than just in the corners, in order to extend the possible range.
This is currently not supported and will require some refactoring of the system since the game is currently generating the field based on the four anchors, which would need to be changed to a system where it is told where the four corner anchors are.
Furthermore the Pozyx sensors use the TWR technique for positioning, as described in \autoref{pozyx:TWR}.
As the field gets larger, there will still only be a single master tag, meaning that some of the slave tags that are too far away from the master tag may not be able to report their position.
Additionally, this may lead to some increasingly irregularly shaped playing fields, which may require another algorithm to generate the goal positions.
