\section{Reflecting on the process}
The process used during the development of this project was described in \autoref{sprint1:ourprocess}, combining elements of Scrum and Essence.
Development was segregated into five sprints, defining the different iterations of the system.
These sprints served as a guideline for when to have sprint retrospectives and enact changes in the process.
Each retrospective would cause slight alterations to the process.
\\\\
Initially, tasks were estimated based on planning poker using story points.
However, we found it difficult to properly estimate a lot of the tasks, as this project concerned itself with an area of expertise we were not familiar with.
This led to the inclusion of an Essence based version of estimating in sprint 1, as described in \autoref{sprint1:retrospective}.
This version of estimation based on reward and time to complete the task was also phased out, as described in \autoref{sec:sprint2-conclusion}, due to it seeming biased towards shorter tasks and estimation not being accurate.
This led to the method of simply having a group-based discussion about the DoD of a task, and ensuring everyone understood it.
This DoD-based task assignment worked well for the final sprints.
It avoided spending unnecessary time on planning poker on top of discussing tasks and made the developers more productive.
\\\\
Another aspect improved by the retrospectives was the stand-up meetings.
At the beginning of the project these tended to be unfocused and not provide the necessary information, and were not conducted daily but rather twice a week.
This led to the introduction of a certain set of questions, defined in \autoref{sec:sprint2-conclusion}, which helped to ensure the developers stayed focused and provided the necessary information to facilitate productivity.
Along with the questions, the stand-up meetings were changed to be conducted daily, which further facilitated the developers staying focused.
\\\\
\subsection{Evaluation of Essence}\label{evaluationessence}
One of the main things that was used from Essence was the roles.
Essence includes things like potential and filters, which are about finding potential components used for the solution, and then narrowing down to a few concrete components, for the pre-project, and other pre-project tools.
These were not introduced until the pre-project phase of the semester had passed, and were deemed less meaningful to use at these stages.\\
Additionally, since not all group members took the software innovation course where Essence was presented, it took a lot of work to relay the information since the remaining group members would essentially have to learn the curriculum of another course on top of the ones they were already following.
\\\\
The primary addition to roles in Essence is the addition of the child role.
The child role was supposed to be a fluctuating role which allowed group members to come with feedback and encourage divergent thinking.\\
However, since many of the pre-project parts of Essence, related to analysing the problem domain, were skipped, the child role did not make a lot of sense and quickly ended up not being used for the intended purpose but rather to suggest silly ideas.
\\\\
Had we known about all parts of Essence before starting the project, there is a large possibility that it would have been significantly more useful, and that going through the pre-project aspects would have helped in increasing the quality of the final system.
These aspects will not be described in further depth, as we chose not to use them in our project.
This assumption is based on the theory that having a clearer idea about the direction and encouraging more in-depth discussions before making choices would have resulted in different choices along the way.
\\\\
One of the primary motivators behind using Essence over an agile approach is the focus on innovation, where it is encouraged to shift direction along the way, while still staying within a pre-defined problem domain.
While we attempted to keep this approach in mind, it was difficult due to the strict time constraint that prohibited us from making major changes in the focus, since we would still have to consider the study regulations and whether we could change focus and still be done in time for every possible pivot.
An example of where we decided to pivot despite the cost involved was in \autoref{sec:update-network-protocol} where the networking protocol was almost entirely reworked from being purely UDP to being a mix of UDP and TCP.
Another part of the project where the use of Essence shines through is the sprint-based structure, which allowed for easy pivoting without requiring extensive rewriting of the report.
This structure also shows how the project has progressed and changed throughout the semester, rather than just presenting the final version.
One of the bad experiences about the structure of the report is that it is more difficult to insert sections into the report.
If something is completed in a sprint, but is written about later, it can be difficult to decide where to insert the section - likewise things like the network protocol may be difficult to get an overview of, since the details about it are spread out across the entire report. 
We also suspect that the coherency of the report might be lower than if the standardized report structure were to be used.
Ultimately, it is doubtful whether Essence is something that we will use in future projects, since it did not seem to add anything useful compared to previous semesters where a traditional agile was used.