\section{UDP and TCP for the protocol}
When the socket type to be used for the program was initially chosen in \autoref{sec:sprint1-networking}, a large emphasis was put on the rate at which we could send messages.
The sending rate also affected a lot of the other decisions made for the networking protocol, such as keeping messages as short as possible.
The goal was to facilitate sending as many messages as possible.
However, once testing began, it became obvious that this might not have been the fully correct approach.
While running the host with mock data, it was possible to more or less crash our connections.
\\\\
In order to fix this, we implemented some message throttling, described in \autoref{sec:sprint5-transmissionlimit}, based on the maximum amount of updates from the available Pozyx system.
Keeping this limiting of messages in mind, the aspect of sending messages rapidly loses some of its importance, leading to some of our earlier considerations not being optimal.
This might have led us to implement a TCP aspect earlier in the project, avoiding having to spend some of the time attempting to implement UDP acknowledgments.