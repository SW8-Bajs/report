\section{Camera handling}
This section will explain how the camera in the game is handled.
Since the playing field can have a varying size the camera needs to be able to fit the whole playing field on the screen so that it can be rendered.
Another problem is the usage of VR headsets that allow the player to turn the camera's view by looking in different directions.
This causes the playing field to go out of the player's view which is a problem since they can then no longer see their position.
\\\\
The way this is solved is by always having the playing field and its attached game objects face in the direction of the camera.
This is achieved by giving the playing field the same rotation as the camera.
By doing this, the player will always see the playing field in the same position in the 3D space no matter which direction they look.
But this solution is not enough since the playing field will not have the same position in the 3D space as the camera.
It must be a certain distance away from the camera, based on its width and height, to allow it to be rendered. 
Because of this, a new position must be calculated for the playing field in relation to the cameras direction. 
\begin{lstlisting}[caption={Calculating and setting the position and rotation of the playing field}, captionpos=b,language=C,label={lst:camera_position}]
    newPosition = playingFieldTransform.transform.position + playingFieldTransform.transform.forward * distanceFromCamera;
    transform.position = newPosition;
    transform.rotation = mainCamera.transform.localRotation; 
\end{lstlisting}
The new position is calculated by adding the playing field's initial position relative to the camera \texttt{playingFieldTransform.transform.position} with an offset vector.
The offset vector is calculated by multiplying \texttt{playingFieldTransform.transform.forward}, which is a normalized vector pointing in the camera's looking direction, with a certain distance \texttt{distanceFromCamera} that we want the field to be from the camera.
