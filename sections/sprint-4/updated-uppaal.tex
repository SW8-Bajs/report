\section{Updating the \uppaal model}
As mentioned in \autoref{sec:update-network-protocol} the approach to handling the network data has changed from being purely UDP to a combination of UDP and TCP.
This led to the previous \uppaal model specified in \autoref{sec:sprint3-uppaal} being outdated, and having to be updated to fit the new requirements.
\\
Additionally, the previous model focused mostly on the flow of the program and did not properly take the network part of the system into account.
In this new model, the system will consist of four parts: The host, which is meant to represent the Python program that the game facilitator uses, an arbitrary number of clients who have connected to the game, and the UDP and TCP connection between those.
We created a simple application-level model to get an overview of the system and a more extensive model that also models parts of the TCP communication between the host and the client.
The application-level model will be described first in order to get a general overview of the flow of the system.

\subsection{Application-level \uppaal model}
There are two templates in this model which are \uppTemp{Host} and \uppTemp{Client}.

\subsection*{Host}
The \uppTemp{Host} template can be seen on \autoref{fig:sprint4-uppaal-simplified-host}. 
The \uppProc{Host} starts in the \uppLoc{Lobby} and waits for all clients to connect.
Everytime a \uppProc{Client} connects the \uppProc{Host} transitions to  \uppLoc{Connecting}.
It then either acknowledges the connection or waits for the  \uppProc{Client} to timeout.
When the connection is acknowledged it transitions to \uppLoc{Connected}, sends configuration data to the \uppProc{Client} and transitions back to the lobby.
When all clients have connected the \uppProc{Host} transitions to \uppLoc{InGame} and signals to all of the connected clients that the game is starting through \uppOut{TCP_start_game}.
While \uppLoc{InGame} the \uppProc{Host} can either send tag position through UDP, send goal position and goal scored through TCP or end the game through TCP and transition back to \uppLoc{Lobby}.
\begin{figure}[h]
    \centering
    \includegraphics[width=1\linewidth]{sprint4/simplified-host.png}
    \caption{The \uppTemp{Host} template of the abstracted \uppaal model.}
    \label{fig:sprint4-uppaal-simplified-host}
\end{figure}

\subsection*{Client}
The \uppTemp{Client} template can be seen on  \autoref{fig:sprint4-uppaal-simplified-client}.
The \uppProc{Client} starts in the \uppLoc{Initial} location, it then tries to connect and transitions to \uppLoc{Connecting}.
While in \uppLoc{Connecting} it waits for an acknowledgement from the \uppProc{Host}.
If it does not receive an acknowledgment before the timeout limit it outputs a \uppOut{TCP_timeout} and transitions back to \uppLoc{Initial}.
Otherwise, if the \uppProc{Client} receives an acknowledgement from the \uppProc{Host}, it transitions to \uppLoc{Setup} and waits for the \uppProc{Host} to send configuration data.
When the \uppProc{Client} receives the configuration data it transitions to \uppLoc{Lobby} and then waits for the game to start.
When the \uppProc{Host} starts the game, all clients that are in \uppLoc{Lobby} transition to \uppLoc{InGame}.
From there, the \uppProc{Client} waits for the \uppProc{Host} to send packets.
It can receive either a TCP or UDP packet, which makes it transition to \uppLoc{InGame} again, or it can receive a packet that the game has ended which makes it transition to \uppLoc{Initial}.
\begin{figure}[h]
    \centering
    \includegraphics[width=1\linewidth]{sprint4/simplified-client.png}
    \caption{The \uppTemp{Client} template of the simplified \uppaal model.}
    \label{fig:sprint4-uppaal-simplified-client}
\end{figure}

\subsection{Full \uppaal model}
The more extensive \uppaal model will now be discussed and the templates in it will be described, starting with the system declarations of the model.

\subsection*{System declarations}
\begin{uppaalcode}[caption={System declarations}, label={lst:uppaal4:systemdecl},captionpos=b]
    system UDP_host, TCP_host, TCP_client, Host, Client;
\end{uppaalcode}
The project consists of five templates: \uppTemp{Host}, \uppTemp{Client}, \uppTemp{TCP_client}, \uppTemp{TCP_host}, \uppTemp{UDP_host} seen in \autoref{lst:uppaal4:systemdecl}.
As mentioned above, the \uppTemp{Host} and \uppTemp{Client} templates are meant to represent the interface that users interact with when using the system.
In this new version of the model, the inclusion of \uppTemp{TCP_client}s, \uppTemp{TCP_host}s, and a \uppTemp{UDP_host} allows us to model the behavior of the network aspect of the system.

% System declarations
\subsection*{Global declarations}
The global declarations (seen on \autoref{lst:uppaal4:globaldecl}) for the system have two purposes: Setting game and simulation specific options, such as how many clients should be instantiated, or how long the TCP client should wait for a response before declaring the connection as timed out.
\\
To facilitate data transfer between the templates, the model makes use of global buffers.
Each place in the \uppVar{TCP_buffer} array corresponds to a buffer related to a \uppTemp{Client} or \uppTemp{Host} process so that \uppProc(Client(0)) will read from index 0 of the array, \uppProc(Client(1)) will read from index 1 and so on.
The \uppVar{host} variable is saved as an alias to know which space in the buffers is reserved for packets from clients to the host.
Since the \uppVar{UDP_buffer} is supposed to correspond to a multicast, it is simply saved as an integer instead, which will hold the information that any client listening to the multicast will receive.
Finally, a series of channels are defined.
The channels are used to synchronize between the different processes and will be further explained when going into details about each template.

\begin{uppaalcode}[caption={Global declarations}, label={lst:uppaal4:globaldecl},captionpos=b]
const int number_of_clients = 4;
const int time_limit = 2;
const int host = number_of_clients;

typedef int[0, number_of_clients - 1] id_t;

// Each TCP instance has their own entry in the TCP buffer. The clients have the entry at index = their id and the host has the last index.
int TCP_buffer[number_of_clients + 1];
int UDP_buffer;
broadcast chan read_UDP_buffer;
chan tcp_sync[number_of_clients], ack_sync[number_of_clients], ack[number_of_clients], send_message[number_of_clients], timeout[number_of_clients];

chan TCP_send[number_of_clients], TCP_connect[number_of_clients], TCP_connected[number_of_clients], TCP_client_connected, read_TCP_buffer[number_of_clients];

chan UDP_start, UDP_send;
\end{uppaalcode}

% Host
\subsection*{Host template}
The \uppTemp{Host} has three variables that are declared in \autoref{lst:sprint-4-host-code} which are:
\uppVar{connected_clients} which it uses to keep track of how many clients have connected, \uppVar{local_TCP_buffer} which is where the content of the global TCP buffer is copied into and \uppVar{i} which it uses to keep track of how many packets it has sent. 
The template seen in \autoref{fig:sprint4-uppaal-host} can be divided into three sections.
In the first section the host waits for the clients to connect through TCP.
Every time a \uppTemp{Client} has connected the \uppTemp{Host} receives a \uppIn{TCP_client_connected} synchronization and \uppVar{connected_client} is incremented.
When all of the clients have connected the \uppTemp{Host} moves on to the next section.
In this section the \uppTemp{Host} uses the \uppOut{TCP_send} synchronization to notify each \uppTemp{Client} that the game is starting through TCP.
Finally the \uppTemp{Host} is \uppLoc{InGame} and continuously sends packets to the clients.
This is done through either \uppOut{TCP_send} or \uppOut{UDP_send} which are the TCP and UDP channels respectively.
\begin{figure}[h]
    \centering
    \includegraphics[width=1\linewidth]{sprint4/Host.png}
    \caption{The Host template of the \uppaal model.}
    \label{fig:sprint4-uppaal-host}
\end{figure}

\begin{uppaalcode}[caption={local Host declarations}, label={lst:sprint-4-host-code},captionpos=b]
int connected_clients = 0;
int local_TCP_buffer;
int i = 0;
\end{uppaalcode}


% Client
\subsection*{Client template}
Like the \uppTemp{Host} template, the \uppTemp{Client} has a local declaration for saving the data it reads from the global buffer.
Since it is desirable to have a variable amount of instances, the template has an \uppType{id_t} parameter called \uppVar{id}, to allow instantiating multiple client processes.
The \uppTemp{Client} has four named locations, as seen on \autoref{fig:sprint4-uppaal-client}, which correspond to the states that the actual implementation can be in.
First off, the process starts in an \uppLoc{Initial} location, which then uses the \uppOut{TCP_connect[id]} to make the \uppTemp{TCP_client(id)} start connecting through TCP and transition to \uppLoc{Connecting}.
The \uppProc{Client(id)} receives \uppIn{TCP_connected[id]} synchronization when the \uppProc{TCP_client(i)} has connected and it then transitions to the \uppLoc{Lobby} location.
Everytime it receives a \uppIn{read_TCP_buffer[id]} it means that the \uppTemp{Host} sent a packet.
It then transitions to an in-between location and then either transitions back to \uppLoc{lobby} or to \uppLoc{InGame} depending on what packet is received.
The in-between location is a committed location.
This is done so that the \uppProc{Client(id)} has to transition from it immediately instead of waiting.
Finally, there is the \uppLoc{InGame} location, which is where the client will be located most of the time during the simulation.
This indicates that the client is now in a game, and can continuously receive information about the game state through the \uppIn{read_UPD_buffer} and \uppIn{read_TCP_buffer[id]} synchronizations.
\begin{figure}[h]
    \centering
    \includegraphics[width=1\linewidth]{sprint4/Client.png}
    \caption{The Client template of the \uppaal model.}
    \label{fig:sprint4-uppaal-client}
\end{figure}

% TCP
\subsection*{TCP host and client templates}
The \uppTemp{TCP_Host} and \uppTemp{TCP_Client} were designed such that there will be a single process of both respectively for each client process that exists.
This was designed to model the connection that is established between the host and a client.
Like in the \uppTemp{Client}, this happens by having a \uppVar{clientId} as a parameter for when the process is instantiated.
As seen on \autoref{fig:sprint4-uppaal-tcp-client} and \autoref{fig:sprint4-uppaal-tcp-host}, the TCP templates will stay in a \uppLoc{Start} location until \uppProc{TCP_client} receives an \uppIn{TCP_connect[id]} synchronization from a \uppProc{Client}.
The \uppTemp{TCP_Host} and \uppTemp{TCP_Client} will then begin to initialize a 3-way handshake to ensure that both parties are ready for transferring data between them.
Once the handshake has been completed, the \uppTemp{TCP_Host} will send a packet to the \uppTemp{Host} to inform that a new player has been connected via the \uppOut{TCP_client_connected} synchronization, and then enter an \uppLoc{Idle} location to indicate that it is ready to send packets.
Likewise, in the \uppTemp{TCP_Client}, after the handshake is completed it will inform the \uppTemp{Client} that a connection has been successfully established through the \uppOut{TCP_connected[id]} synchronization, and transition to the \uppLoc{Idle} location where it is ready to receive packets.

\begin{figure}[h]
    \centering
    \includegraphics[width=1\linewidth]{sprint4/TCP_client.png}
    \caption{The TCP\_client template of the \uppaal model.}
    \label{fig:sprint4-uppaal-tcp-client}
\end{figure}

\begin{figure}[h]
    \centering
    \includegraphics[width=1\linewidth]{sprint4/TCP_host.png}
    \caption{The TCP\_host template of the \uppaal model.}
    \label{fig:sprint4-uppaal-tcp-host}
\end{figure}

% UDP
\subsection*{UDP host templates}
The final template seen in \autoref{fig:sprint4-uppaal-udp-host} is the \uppTemp{UDP_host}.
It is very simple as it waits until the \uppOut{Host} communicates that the game has started through \uppIn{UDP_start}.
When the \uppTemp{UDP_host} has started it waits for a \uppIn{UDP_send} synchronization from the host and then notifies the clients through the \uppOut{read_UDP_buffer} synchronization that they should read the global buffer.
\begin{figure}[h]
    \centering
    \includegraphics{sprint4/UDP_host.png}
    \caption{The UPD\_host template of the \uppaal model.}
    \label{fig:sprint4-uppaal-udp-host}
\end{figure}

% Trace example of the system
\subsection{Verification of the models}
In \uppaal it is also possible to perform verifications on the model to ensure that some properties hold true for the model.
This is done by creating a query that the \uppaal engine can then verify on the model.
It is then possible to get a trace of that query that either proves or disproves the query.
To demonstrate this a query was verified on the application-level model and the full model.
The property that was verified was \uppEF{Host.InGame and forall(i:id_t) Client(i).InGame}.
This query checks if there exists a trace where at some point the property holds true that the host is in the \uppLoc{InGame} location and all of the clients are in the \uppLoc{InGame} locations.
The traces that we get from the verification is the shortest trace that satisfies the properties defined.

\subsubsection{Verification of the application-level model}
The property held true for the model an the trace that was produced can be seen on \autoref{fig:sprint4-verification-application-level}.
As can be seen on the trace, all of the clients go through the process of connecting without timing out.
When all of the clients have connected, the \uppProc{Host} starts the game through the \uppSync{TCP_start_game} synchronization and the host and all of the clients transition to the \uppLoc{InGame} location.
\begin{figure}[H]
    \centering
    \includegraphics[width=0.8\linewidth]{sprint4/application-level-verification.png}
    \caption{Verification trace of the application-level model.}
    \label{fig:sprint4-verification-application-level}
\end{figure}

\subsubsection{Verification of the Full model}
The property also held true for the full model and the trace can be seen on \autoref{fig:sprint4-verification-full-model}.
As can be seen in the trace all of the clients connect through the three-way handshake.
All of the clients successfully transition to \uppLoc{Lobby} without any timeouts.
When all of the clients have connected the host makes all of the clients transition to \uppLoc{InGame} through TCP.
After all of the clients have transitioned to the \uppLoc{InGame} location the host also transitions to \uppLoc{InGame} and the game starts.
\begin{figure}[H]
    \centering
    \includegraphics[width=0.8\linewidth]{sprint4/full-model-verification.png}
    \caption{Verification trace of the full model.}
    \label{fig:sprint4-verification-full-model}
\end{figure}

Both of the traces end in the same state, but there is a difference in how they get there.
While the application-level model has simplified communication between the host and the client the full model trace also expresses the steps that are done in the TCP communication.
It models how the threeway communication works and how TCP messages are sent between the host and the clients to ensure that the clients and the host receive the messages sent.

\subsection{Conclusion}
The models introduced in this section were mostly used to create an overview of the network protocol and the flow of the game.
All the verification done on the models was done to verify that the models were behaving as expected and not to test the actual system.
It might have been useful to use to models to verify the actual implementation of the network protocol and the flow of the application, but this was deemed to not be a priority for the project.
So the models were only created to function as a visual representation of the application and the network protocol and to ensure that we had a unified understanding of how the flow of the system should be.