\section{Receiving the information}
With the networking protocol in place, let us take a look at how the data is received on the mobile clients.
The \texttt{in-game} scene in Unity has an \texttt{empty object} which handles receiving data from the UDP and TCP clients.
Having two separate clients allows them to run concurrently, so the UDP client does not block the TCP client or vice-versa.

\subsection*{UDP client}
Handling the datagrams on the UDP client is simple as it can currently only receive a message with type 0, which is to a message containing a position for a Pozyx tag.
The datagram is received in the format \textit{0xYYYYXXXXIISSTT} where 0x indicates hex and the rest of the identifiers are:
\begin{itemize}
    \item Y - Y position of tag
    \item X - X position of tag
    \item I - Id of player
    \item S - Timestamp
    \item T - Package type
\end{itemize}

When this datagram is received, we start by parsing the hexadecimal to a long datatype.
Since one hexadecimal corresponds to four bits we can get the type of the message, which is the last two hex values of the datagram by typecasting it to a byte.
As seen on line 13 in \autoref{lst:readingudpdatagram}.

\begin{lstlisting}[caption={Processing datagrams in UDP client}, captionpos=b,language=C,label={lst:readingudpdatagram}]
private void DatagramHandler(string datagramMessage)
{
    // Remove 0x from string before parsing
    if(datagramMessage.ToLower().StartsWith("0x"))
    {
        datagramMessage = datagramMessage.Remove(0, 2);
    }

    long data;
    byte type;
    if (long.TryParse(datagramMessage, System.Globalization.NumberStyles.HexNumber, System.Globalization.CultureInfo.InvariantCulture, out data))
    {
        type = (byte)data;|\label{line:decode-type}|
        switch (type)
        {
            case 0:
                UpdatePlayerData(data);
                break;

        }
    }
    else
    {
        Debug.LogError("Network data could not be parsed");
    }
}
\end{lstlisting}

The next step in processing the data is calling the appropriate function as seen in the switch, in this case we will take a look at the \texttt{UpdatePlayerData}.

\begin{lstlisting}[caption={Updating player data in UDP client}, captionpos=b,language=C,label={lst:updateplayerdata}]
private void UpdatePlayerData(long data)
{
    // bitshifting the hex string and typecasting to byte to get the values.
    // see network format in the report for more detail
    byte time = (byte)(data >> 8);
    byte id = (byte)(data >> 16);
    ushort x = (ushort)(data >> 24);
    ushort y = (ushort)(data >> 40);


    if (CheckTimestamp(time))
    {
        if (id == 0)
        {
            gameStateHandler.ballPosition.x = x;
            gameStateHandler.ballPosition.y = y;
        }
        else
        {
            // Player id starts at 1 while the playerposition array is 0 indexed. Decrementing id so that they line up.
            id--;
            gameStateHandler.playerPositions[id].x = x;
            gameStateHandler.playerPositions[id].y = y;
        }
    }
    
}
\end{lstlisting}

The next step in processing the data is to read the actual content of the datagram sent, in this step we utilize that the fact that all messages have a size corresponding to a type in C\#.
We continue to read the datagram from right to left using the right bitshift operation.
The first part we want to read from the message is the timestamp, which we know from the network format (\autoref{app:network}) has a size of 8 bits, which corresponds to the byte type in C\#.
We know that the type of the package is 8 bits, we can therefore remove the type and get the timestamp to the back of the datagram by right bitshifting it by 8 bits as can be seen on line 5 in \autoref{lst:updateplayerdata} 
We continue doing this for each message in the datagram where we start by right shifting the datagram by the combined size of the previously read messages and then typecasting it to a variable type with the same size as the message that is being read.
\subsection*{TCP client}
The principle behind the TCP client is much like the UDP client, where the type is read as a byte and then a switch statement ensures that the data is handled by the correct function.
However, since TCP is sent as a stream, all packages have been prefixed with a numerical values to indicate the length of each package.
This means that the client cannot simply wait for a new message, read it and act according to the type.
Instead, the client will read the first two bytes of the stream and then read the incoming amount of bytes indicated by this numerical value, allowing the program to handle packages of varying length.
