\section{Updating the networking protocol}
During implementation of the networking protocol outlined in \autoref{sec:sprint3-uppaal}, some problems were encountered.
Section \ref{sec:sprint3-uppaal} defines the way the protocol was envisioned to work based on the choice made in \autoref{sec:sprint1-networking}, where UDP sockets were selected as the type to be used for this project.
With the unreliability of UDP, and the importance of certain data being transmitted to players, it would be necessary to implement an acknowledgment from clients that they had received crucial data relating to the setup of the game, as defined in \autoref{sec:sprint3-uppaal}.
During implementation of this, issues were encountered when sending the acknowledgment message from the client to the host.
As the sockets should make use of multicasting, as defined in \autoref{sec:sprint1-udptransmission}, it seemed as though the clients should just be able to send messages to the same multicast group, and have the host receive that message, with a type that specified that the message was an acknowledgment that could be either positive or negative, structured in the same way as described in \autoref{subsec:data-format}.
However, the host had issues receiving the messages.
This prompted a discussion regarding how to remedy this issue, as a new approach seemed necessary.
Three main alternatives were considered:
\begin{itemize}
    \item Split the networking into two distinct phases - setup and in-game data
    \item Expand the messages being sent to include the crucial data on all messages sent
    \item Introduce a TCP element to make use of its reliability for game critical data
\end{itemize}

\subsubsection{Splitting networking into two distinct phases}
This approach aims to create a clear delineation between the crucial data that requires acknowledgment from clients and the data for which acknowledgments are unnecessary.
This was roughly the original approach, as shown on \autoref{fig:uppaal-host-1}, which illustrates how the host would wait in a setup state until acknowledgments had been received from all clients, and then the game would start.
This presents issues in terms of the multicast grouping, in which the host was unable to read messages sent through the group.
This could be remedied by having a separate multicast group in which clients could send acknowledgments, to avoid issues related to multicasting.
While the host was sending out the information, a client could check if it had received the correct information, and if not send out a negative acknowledgment for a while before timing out, to attempt to receive the information again.
Eventually all the clients would be sending positive acknowledgments.
\\\\
However, an issue exists with this approach.
When scoring a goal, the goalzones should be moved.
This would take place during the phase when the game is played, and receiving the information about the location of the new goalzone is crucial, such that all payers know where to score.
As such, the phase split loses some of its purpose, as an acknowledgment would need to be sent from clients during the playing phase.

\subsubsection{Expanding messages}
In addition to the data defined in \autoref{app:network}, crucial data could be appended to all messages sent during play.
This would mean that whenever any message was received, it would include to required information, and no acknowledgment would be necessary as the clients are guaranteed to receive some messages if they are continuously sent.
This would have the detriment of increasing the size of each individual package.
While the packages are currently not large enough to where this is likely to cause a problem, creating minimal packages is still a worthwhile goal in order to increase the speed of transmission and decoding.

\subsubsection{Introducing TCP}
On top of the currently described UDP approach to sending messages to ensure recent data is used for positioning while the game is played, a TCP socket connection between the host and clients could be implemented.
As described in \autoref{sec:sprint1-networking}, TCP includes reliability.
This reliability ensures that every message sent will be received by the client, through built in acknowledgments.
This would remove the need for clients sending UDP acknowledgments for game crucial data.
Because of the reliability, and needing established connections, TCP messages will be slower to transmit.
As one of the focuses for the project is creating a networking solution capable of continuously updating positional data as quickly as possible, this is not an ideal solution for transmitting the data when the game is being played.
As such, a combination is proposed - TCP for crucial data, and UDP for positional data.
The issue described above relating to goalzones can also be remedied through a combination.
Rather than send location for the new goalzones while the game was being played and waiting for an acknowledgment, this data could be sent on the TCP connection as well, to ensure it arrives.

\subsubsection{Final choice of updating}
Based on the three alternatives to remedy the issues encountered, introducing a separate TCP connection between the hosts and the clients to handle crucial data that requires acknowledgments seems like the ideal choice.
It avoids the confusion inherent in the unreliability of UDP requiring manual handling of acknowledgments, and avoids communication back and forth, while allowing fast update times for positional data through the UDP connection.
 
