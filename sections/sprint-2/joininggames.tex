\section{Accessing games on the network}\label{sec:accessonnetwork}
Different methods of transmission were discussed in \autoref{sec:sprint1-udptransmission}, and multicasting was selected as the optimal solution for this project.
In order to make proper use of this in an eventual deployment of the game, it would not make sense for the users to input the IP address to which they want to connect.
As such, a way to access hosted games without this needs to be implemented.
In order to support multiple games being played at the same time, the system also needs to properly make use of multicast groups based on the game being joined.
Clients should only receive data from one specific host in their specific multicast group.
\\\\
To do this, the game must include some form of LAN discoverability.
One way of achieving this is to have clients broadcast a packet to available hosts via the LAN when they are searching for a game.
The hosts then reply, if they are available, with data for a multicast group, and the client can then join that group.
Another method could be to make use of Internet Group Management Protocol (IGMP), which is a communications protocol used on IPv4 networks to establish multicast groups.
The IP address 224.0.0.1 is a notable IPv4 address reserved for IP multicasting, which is the multicast group address for all hosts on the same network \cite{ipv4multicastaddresses}.
All hosts should join that group on start-up, and hosts could then send a packet to the group to signal that they are looking for an available game.
\\\\
Another way would be to have the host continually broadcast packets that it is available for players.
This would, however, lead to the host needing to repeatedly send packets as long as the game it is hosting has available slots for players or has not started.
This would likely be a less elegant solution than having players that are searching for a host sending the packet.
Having the players send a packet to the multicast group address for all clients is the preferable solution, as it will avoid unnecessary overhead caused by broadcast oackets being sent to unrelated machines on the network.
\\\\
Another option is to make use of the Zero Configuration Networking (Zeroconf) protocol.
Zeroconf is a protocol that allows for browsing of available services.
This can be used to facilitate IP multicasting in a way relevant for the game, where the host could make the service available and send the packets to all interested clients.
Zeroconf selects an IP address within a relevant range, and then claims that address if available.
It will then create a name for the host device, such that it is not necessary to remember and type numerical addresses.
Services can then be made available for discovery \cite{zeroconf}.
\\\\
In order to select a game to join as a client, the game would need a game selection screen.
\autoref{fig:prototype:menu} and \autoref{fig:prototype:menuconnected} show the initial prototypes for joining a host based on their IP and waiting for the game to begin.\\
If a lobby were to be implemented, \autoref{fig:prototype:menu} would need to be updated, and there would need to be an extra step before \autoref{fig:prototype:menuconnected} could be shown.
\autoref{fig:sprint2:prototype:menu} and \autoref{fig:sprint2:prototype:lobby} show new prototypes for this purpose.
Choosing a lobby on \autoref{fig:sprint2:prototype:lobby} would lead to \autoref{fig:prototype:menuconnected}.
\begin{figure}[H]
    \centering
    \includegraphics[width=0.6\linewidth]{prototypes/sprint2-menu.png}
    \caption{Prototype of the game menu when players can choose a game to join.}
    \label{fig:sprint2:prototype:menu}
\end{figure}

\begin{figure}[H]
    \centering
    \includegraphics[width=0.6\linewidth]{prototypes/sprint2-lobby.png}
    \caption{Prototype of a lobby menu. Available games are shown as well as the number of players. Players can choose one to join.}
    \label{fig:sprint2:prototype:lobby}
\end{figure}
