\section{Sprint 2 conclusion}

\subsection{Project status}


\subsection{Retrospective on the process}
The process required some unexpected pivots in the second sprint due to the coronavirus, which lead to the university being locked down for a period.
\\
The daily group work was moved from the physical group room to a virtual setting facilitated by Discord.
One of the major differences was that it was previously possible to sit down and do the pair reviews in the morning, which meant that they would go through review quickly.
However, while we tried to still get the pair reviews done first thing in the morning, there was a noticeable delay before the new features were reviewed and merged.\\
Since most of the project work is managed on Jira and completed by individual group members, there was not a noticeable change in the amount of work getting done.
A slight side effect of working with only voice chat has been that it was occasionally difficult to reach the other members when their help was needed since they might be temporarily away from the computer, or too distracted or focused to be paying attention to the voice chat.
This also resulted in some information not getting spread to the entire group, as not everyone may have been listening in while some discussion was happening, which lead to minor confusion in further discussions on the subject.\\\\

Like at the end of sprint 1 (\autoref{sec:sprint1conclusion}), a retrospective was conducted by the anchor to reflect upon how the process is working out.

In addition to discussing the progress of the project, the following points were brought up:

\subsubsection{How does the process with Jira work?}
Currently, it seems like the challenger is the only person adding suggestions to the backlog, after some discussion it turned out that multiple members thought that this was intentional, and did not know that they were also supposed to contribute with their suggestions for the project.
This has been clarified, and all members are now aware that they are fully encouraged to add suggestions to the Jira when they come up with ideas for the project.

\subsubsection{Should we use pair programming more?}
Right now, the only work done in pairs is the pair reviews.
It was decided that utilizing pair programming for larger programming tasks would be beneficial to decrease the amount of time spent on it and allow for more input in how it will be implemented.
Whether or not a task should be done with pair programming will be decided as a part of the task discussions that take place after the daily standups if there are new tasks in the suggested column.

\subsubsection{How do we give better estimates about when a task is done in daily standups?}
Generally, the biggest reason that it is difficult to estimate how much remains of a given task, is that the definitions of done are not precise enough.
To battle this, it was decided to remove the prioritization of tasks where we would assign a reward, cost, and priority.
Instead, the discussion will be about what a good and specific DoD is for the given task to ensure that everyone is on the same page.
The hope with this approach is that discussing the DoD will give new perspectives to a given task and shift the focus from how long it will take to complete the task and instead focus on exactly what needs to be done.
A possible by-product of this shift of focus is that the in-depth discussion will result in discovering new tasks that need to be done.\\

\subsubsection{How do we feel about daily standups?}
Everyone is generally pleased with the daily standups, but there is a tendency to focus more on what has already been done rather than what is currently being worked on.
To prevent this, a new format of the meetings has been proposed:

\begin{itemize}
	\item{What have I been doing? (short)}
	\item{What am I working on now?}
	\item{When will my task be completed?}
	\item{What is my current challenge?}
	\item{Do I need help or reviewers?}
\end{itemize}

After each member has presented these four points, we will go through the new tasks in the suggested column and define a DoD for them.

Finally, we went through all tasks that had been completed in the sprint to ensure that everything that had been implemented was also documented in the report, to ensure that the knowledge is shared between all members of the group.
