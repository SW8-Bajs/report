\section{Test plan}

\subsection{Documentation}
All functions should be documented. 
This is done by writing comments in the code following the standards of the language.
\\\\
For C\#: 
\\
\url{https://docs.microsoft.com/en-us/dotnet/csharp/language-reference/language-specification/documentation-comments}
\\\\
For Python:
\\
\url{https://www.geeksforgeeks.org/python-docstrings/}
\\\\
These comments should explain what the parameters of the function are and a short summary of what the function does.
\\\\
If the function that is being implemented deals with more complex concepts, additional comments should be written in the code to explain what is happening in the function.

\subsection{Testing}
\subsubsection{What to test}
All code that can influence the user's experience should be tested. 
This means that every function that would influence the program negatively if it had an incorrect calculation should be tested to try to avoid this.
\\
All public functions should always be tested.

\subsubsection{How much to test}

\begin{itemize}
    \item Unit tests
        \begin{itemize}
            \item Functions are tested in isolation.
            \item If the function needs data from outside the function it should be manually generated using mocks.
            \item \href{http://softwaretestingfundamentals.com/unit-testing/}{More information about unit testing}
        \end{itemize}
    \item Integration tests
        \begin{itemize}
            \item Testing if different modules integrate correctly.
            \item An example of this is when we are fetching data from the python server into the C\# client.
            \item \href{http://softwaretestingfundamentals.com/integration-testing/}{More information about integration testing}
        \end{itemize}
\end{itemize}

\subsubsection{How much to test}
100\% code coverage is not necessary, but if there are no unit tests for a part of the system it is not enough. 
The focus should be on the quality of tests rather than the number of tests.
\\
All functionality in a function that can affect the output of the function should be tested.
If a bug is fixed and it took more than a couple of minutes to fix it, unit tests should be created for this bug to prevent it from being reintroduced later on.
