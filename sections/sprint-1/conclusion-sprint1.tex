\section{Sprint 1 conclusion}\label{sec:sprint1-conclusion}
This section concludes the preceding chapter on sprint 1.
It will conclude what knowledge was gained based on the sections of the chapter, and discuss a retrospective of the sprint as a whole, and what changes were made to the progress for upcoming sprints.
\subsection{Overview of completed tasks}
\autoref{sec:sprint1-goals} outlined the goals of the sprint.
The sprint focused on exploring the project idea.
Wireframes were constructed in\autoref{sec:sprint1-prototypes} in order to create a more uniform vision of the project.
These generated a vision for how the users would interact with the game, and how the game aspect would look.
\\\\
In \autoref{sec:sprint1-architecture} a rudimentary overall architecture for the project was outlined.
We defined an architecture in which data would be transferred from the Pozyx components to a host and then to the players of the game.
\autoref{sec:sprint1-pozyx} gave an introduction to the Pozyx technology to be used for the tracking aspect of the project.
Several alternatives for finding anchor locations were proposed, and UWB was defined as the best solution.
\autoref{sec:unity-intro} gave an introduction to the game engine to be used for building the project.
\autoref{sec:sprint1-networking} explored the possibilities for communication between the different parts of the architecture, settling on using a custom UDP based solution.
Finally, \autoref{sprint1:experiment} defines an experiment conducted in order to determine the accuracy of the Pozyx system, concluding that the accuracy was acceptable for the use of this project.

\subsection{Retrospective on the process}
Initially, the project made use of a more scrum-inspired process, which consisted of smaller sub-sprints of one week composing the larger sprints defined as sections in this report.
Each week had a backlog of tasks to be completed that week, evaluated on a story point-based system.
During this sprint, this was changed to the process outlined in \ref{sprint1:ourprocess}.
The shorter sprints were phased out, and instead, a more constant backlog was kept, from which to choose tasks.
We held a retrospective on this process at the end of the sprint.
The developers liked not having the smaller sprints and decided to do reviews in pairs going forward.
To ensure pull requests would be reviewed quickly, a rule to check for pull requests every morning was set in place.
The waypoints were assigned to tasks in the process was questioned, as the way it was done lead to some bias towards tasks that took a short amount of time to complete.
It was decided that creating a weight to influence the tasks in a way to avoid this bias was a good idea.