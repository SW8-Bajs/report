\section{Unity introduction}\label{sec:unity-intro}
As defined in FUK\todo{Add ref to project idea section}, this project aims to create a team-building augmented reality game.
This means the project has to have a game component - an application to display the objectives of the game, the play area and the players.
To create this, a game engine can be used, such as \texttt{Unity}.
A game engine is a piece of software that provides creators with the necessary set of features to build games quickly and efficiently\cite{gameengine}.
This means that a game engine is a collection of reusable components, abstracted away from the game developer.
This can include tools to help with, for example, graphics, physics, networking or audio.
These tools would expose certain functionality to a developer to make use of, and hide the specific implementation details for that functionality, ensuring the developer can focus on more pressing issues.
Unity supports the C\# language for development\cite{unitylanguage}.
\\\\
The Unity game engine supports development for different game platforms.
Of particular interest to this project is the support for both \texttt{Android} and \texttt{iOS} devices, as well as \texttt{Google Cardboard}\cite{unityplatforms}.
We chose to use Unity for the development of the game aspect of this project.
This facilitates that a greater amount of time can be spent on the other aspects of the project rather than the low-level details of game development, and it allows for easier inclusion of multiple platforms.