\section{Pozyx}
Pozyx is a hardware/software solution that is used to provide positioning with an accuracy of down to 10 cm \cite{pozyx}.
It makes use of ultra-wideband in combination for positioning, which according to their documentation is more precise and effecient than traditional positioning systems such as WiFi, bluetooth, RFID and GPS.

Since the two major requirements for the positioning in this project are precision and a high update rate to ensure that the players are able to have reliable data available, the Pozyx system seems like a good place to start.

For update rate, the tags support update rates up to 125 Hz for a single tag \cite{pozyx}.

The Creator system from Pozyx is sold with 4 anchors and 5 locatable tags.

\subsection{Finding the location of anchors}
For finding the position of a given tag using the anchors, a trilateration method is used.

This method uses basic geometry to estimate the position by measuring the distance to the anchors of which we know the position.

With this data, it is possible to draw a circle with the given radius.
If we use two anchors, we will have two intersection points which are the possible positions of the tag.
This means that to find a two dimensional location, we will need at least three anchors, which will lead to only a single point where all three circles intersect.

The issue with this approach is that the measurements are not perfect, which will cause the circles to not intersect at exactly one point.

\subsection{Using UWB}
TWR
TDOA
Problems with UWB