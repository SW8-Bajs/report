\pdfbookmark[0]{English title page}{label:titlepage_en}
\aautitlepage{%
  \englishprojectinfo{
    Title of the project%title
  }{%
     Mobility %theme
  }{%
    Spring Semester 2020 %project period
  }{%
    SW805F20 % project group
  }{%
    %list of group members
    Andreas Stenshøj\\
    Daniel Moesgaard Andersen\\
    Frederik Valdemar Schrøder\\
    Jens Petur Tróndarson\\
    Rasmus Bundgaard Eduardsen\\
    Mathias Møller Lybech
  }{%
    %list of supervisors
    Brian Nielsen\\
  }{%
    1 % number of printed copies
  }{%
    May 28th, 2020 % date of completion
  }%
}{%department and address
  \textbf{Department of Computer Science}\\
  Aalborg University\\
  Selma Lagerlöfs Vej 300\\
  9220 Aalborg East, DK\\
  \href{www.cs.aau.dk}{www.cs.aau.dk}
}{% the abstract
Can Pozyx, a highly accurate positioning solution based on UWB, be used to create a virtual reality game that represents the user with real-time positional data in-game?
In this project, we will look into how to obtain positional data, model a network protocol for transferring the data to the players, and create a virtual reality-based game that shows the data.
In developing the project, the pragmatic paradigm presented by Ivan Aaen in his research about Essence will be utilized to encourage innovation and diverse thinking in the development process.
The results show that while Pozyx provides high accuracy, the refresh rate that it provides is not high enough for a real-time experience.
It is concluded that combining Pozyx with a technique like dead reckoning would possibly improve the experience.
}


